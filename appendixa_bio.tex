
\section{Biographical Sketches of Senior Investigators}
\label{sec:bio}
%\textcolor{red}{Provide a biographical sketch (CV) for each senior person (can be the same as for the management review or in original proposal), including those at partner institutions. Senior personnel are all those named on the staffing table, whether or not they receive funding from the CMS project. Provide concise vitae, listing professional and academic essentials and complete contact information. Include a list of up to ten publications that are most pertinent to the project.}
\subsection*{Tadashi Ogitsu, PI}

Lawrence Livermore National Laboratory: Senior Staff Scientist, Quantum Simulation Group, Materials Science Division, Physical and Life Sciences Directorate; 926-422-8511; ogitsu1@llnl.gov

\subsubsection*{Education and Training}

\begin{table}[ht]
\centering
\begin{tabular}{llll} 
University of Tsukuba  & Energy Conversion Engineering & B.A. & 1989 \\ 
University of Tsukuba & Materials Sciences & M.S. & 1991 \\
University of Tsukuba & Materials Sciences & PhD. & 1994 \\
\end{tabular}
\end{table}

\subsubsection*{Research and Professional Experience}

\begin{table}[ht]
\centering
\begin{tabular}{ll} 
Senior Staff Scientist in QSG, Lawrence Livermore National Laboratory & 2016-present\\
Staff Scientist in QSG, Lawrence Livermore National Laboratory & 2001-2016 \\
Postdoctral fellow, University of Illinois at Urbana-Champaign & 1998-2001 \\
Research Associate, Institute for Solid State Physics, University of Tokyo & 1994-1998 \\
\end{tabular}
\end{table}

\subsubsection*{Publications}
\begin{enumerate}
    \item T. Ogitsu, A. Fernandez-Pañella, S. Hamel, A.A. Correa, D. Prendergast, C.D. Pemmaraju, Y. Ping, “Ab initio modeling of nonequilibrium electron-ion dynamics of iron in the warm dense matter regime” {\it Phys. Rev. B} {\bf 97}, 214203 (2018).
    \item B. Cho, T. Ogitsu, K. Engelhorn, A. Correa, Y. Ping, J. Lee, L. Bae, D. Prendergast, R. Falcone, P. Heimann, “Measurement of Electron-Ion Relaxation in Warm Dense Copper” {\it Sci. Rep.} {\bf 6}, 18843 (2016).
    \item T. Ogitsu, Y. Ping, A. Correa, B.-i. Cho, P. Heimann, E. Schwegler, J. Cao, G.W. Collins, “Ballistic electron transport in non-equilibrium warm dense gold” {\it High Energy Density Physics} {\bf 8}, 303 (2012).
    \item B.I. Cho, P.A. Heimann, K. Engelhorn, J. Feng, T.E. Glover, M.P. Hertlein, T. Ogitsu, C.P. Weber, A.A. Correa, R.W. Falcone, “Picosecond Single-Shot X-ray Absorption Spectroscopy for Warm and Dense Matter” {\it Sync. Rad. News} {\bf 25}, 12 (2012).
    \item B.I. Cho, K. Engelhorn, A.A. Correa, T. Ogitsu, C.P. Weber, H.J. Lee, J. Feng, P.A. Ni, Y. Ping, A.J. Nelson, D. Prendergast, R.W. Lee, R.W. Falcone, P.A. Heimann, “Electronic Structure of Warm Dense Copper Studied by Ultrafast X-Ray Absorption Spectroscopy” {\it Phys. Rev. Lett.} {\bf 106}, 167601 (2011).
    \item Y. Ping, A.A. Correa, T. Ogitsu, E. Draeger, E. Schwegler, T. Ao, K. Widmann, D.F. Price, E. Lee, H. Tam, P.T. Springer, D. Hanson, I. Koslow, D. Prendergast, G. Collins, A. Ng, “Warm dense matter created by isochoric laser heating” {\it High Energy Density Physics} {\bf 6}, 246 (2011).
    \item E.D. Murray, S. Fahy, D. Prendergast, T. Ogitsu, D.M. Fritz, D.A. Reis, “Phonon dispersion relations and softening in photoexcited bismuth from first principles” {\it Phys. Rev. B} {\bf 75}, 184301 (2007).
    \item Y. Ping, D. Hanson, I. Koslow, T. Ogitsu, D. Prendergast, E. Schwegler, G. Collins, A. Ng, “Broadband dielectric function of nonequilibrium warm dense gold” {\it Phys. Rev. Lett.} {\bf 96}, 255003 (2006).
    \item S.A. Bonev, E. Schwegler, T. Ogitsu, G. Galli, “A quantum fluid of metallic hydrogen suggested by first-principles calculations” {\it Nature} {\bf 431}, 669 (2004).
    \item H. Kitamura, S. Tsuneyuki, T. Ogitsu, T. Miyake, “Quantum distribution of protons in solid molecular hydrogen at megabar pressures” {\it Nature} {\bf 404}, 259 (2000).
\end{enumerate}

\subsubsection*{Synergestic Activities}
\begin{itemize}
    \item 2019-present. Co-author of Co-design center proposal, which was invited to submit a Position Paper, "Co-designing from Atoms to Architectures" led by S. Shankar1 of Harvard
    \item 2003-2020: Tri-lab (LBNL/SLAC/LLNL) collaboration on the study of electron-ion non-equilibrium systems in warm dense matter condition created by femto-second laser. Members: Roger Falcone (UC Berkeley), David Prendergast (LBNL), Das Pemmaraju (SLAC), Phil Heimann (SLAC), Alfredo Correa (LLNL), Amalia Fernandez-Panella (LLNL), Yuan Ping (LLNL).  See publication 1-8.
    \item 2001-present: a user of high-performance ab-initio molecular dynamics codes, GP and Qbox, developed by Francois Gygi of U. C. Davis. This often involved benchmarking on particular high performance computer systems introduced at Lawrence Livermore National Laboratory through Grand Challenge Supercomputing Program.  
    \item 1994-1998: A staff member for theory facility (supercomputer center) of Materials Design and Characterization Lab, Institute for Solid State Physics, University of Tokyo. Co-managed the supercomputer center. Note: the center operated the 15th fastest supercomputer in the world (https://www.top500.org/list/1995/11/). Served for High-Performance-Fortran/Japan (HPF/JA) language specification development activity led by Dr. Hajime Miyoshi (Jack Dongarra of Japan). http://www.hpfpc.org/index-E.html
    \item 1994-1998: Implemented a parallel Feynmann path-integral density-functional-theory molecular dynamics algorithm to his ab-initio code, performed simulations of quantum-mechanical solid hydrogen on CP-PACS, which was the world fastest supercomputer in 1996 (https://www.top500.org/list/1996/11/), and the results published as Nature 404, 259 (2000): it was explicitly an architecture-aware code (remote direct memory access/pseudo vectorization etc)
\end{itemize}
\clearpage


\subsection*{Alfredo A. Correa, Software Manager}
Lawrence Livermore National Laboratory: Staff Scientist, Quantum Simulation Group, Materials Science Division, Physical and Life Sciences Directorate

\subsubsection*{Education and Training}

\begin{table}[ht]
    \centering
\begin{tabular}{llll} 
Instituto Balseiro, Universidad de Cuyo& Physics & B.A. & 2001 \\ 
University of California Berkeley & Physics & PhD & 2008 \\
\end{tabular}
\end{table}

\subsubsection*{Research and Professional Experience}
\begin{table}[ht]
\centering
\begin{tabular}{ll} 
Staff Scientist in QSG, Lawrence Livermore National Laboratory & 2012-present\\
Thrust 1 Leader, EDDE EFRC & 2016-2018 \\
Lawrence Fellow, Lawrence Livermore National Laboratory & 2009-2012 \\
\end{tabular}
\end{table}

\subsubsection*{Publications}
\begin{enumerate}
    \item T. Ogitsu, A. Fernandez-Pañella, S. Hamel, A.A. Correa, D. Prendergast, C.D. Pemmaraju, Y. Ping, “Ab initio modeling of nonequilibrium electron-ion dynamics of iron in the warm dense matter regime” {\it Phys. Rev. B} {\bf 97}, 214203 (2018).
    \item T. Ogitsu, Y. Ping, A. Correa, B.-i. Cho, P. Heimann, E. Schwegler, J. Cao, G.W. Collins, “Ballistic electron transport in non-equilibrium warm dense gold” {\it High Energy Density Physics} {\bf 8}, 303 (2012).
    \item B.I. Cho, K. Engelhorn, A.A. Correa, T. Ogitsu, C.P. Weber, H.J. Lee, J. Feng, P.A. Ni, Y. Ping, A.J. Nelson, D. Prendergast, R.W. Lee, R.W. Falcone, P.A. Heimann, “Electronic Structure of Warm Dense Copper Studied by Ultrafast X-Ray Absorption Spectroscopy” {\it Phys. Rev. Lett.} {\bf 106}, 167601 (2011).
    \item A. Tamm, M. Caro, A. Caro, G. Samolyuk, M. Klintenberg and A. A. Correa, “Langevin dynamics with spatial correlations as a model for electron-phonon coupling,” {\it Physical Review Letters} {\bf 120}, 185501 (2018).
    \item R. Ullah, E. Artacho and A. A. Correa, “Core electrons in the electronic stopping of heavy ions,” {\it Physical Review Letters} {\bf 121}, 116401 (2018).
    \item Erik W. Draeger, Xavier Andrade, John A. Gunnels, Abhinav Bhatele, André Schleife, Alfredo A. Correa, “Massively parallel first-principles simulation of electron dynamics in materials,” {\it Journal of Parallel and Distributed Computing} {\bf 106}, 205 (2017).
    \item 7.	Valerio Rizzi, Tchavdar N. Todorov, Jorge Kohanoff and Alfredo A. Correa, “Electron-phonon thermalization in a scalable method for real-time quantum dynamical simulations,” {\it Physical Review B} {\bf 93}, 024306 (2016).
    \item André Schleife, Yosuke Kanai and Alfredo A. Correa, “Accurate atomistic first-principles calculations of electronic stopping,” {\it Physical Review B} {\bf 91}, 014306 (2015).
    \item André Schleife, Erik Draeger, Yosuke Kanai, Alfredo A. Correa, “Plane-wave Pseudopotential Implementation of Explicit Integrators for Time-dependent Kohn-Sham Equations in Large-scale Simulations,” {\it The Journal of Chemical Physics} {\bf 137}, 22A546 (2012).
    \item Alfredo A. Correa, Jorge Kohanoff, Emilio Artacho, Daniel Sánchez-Portal and Alfredo Caro, “Nonadiabatic Forces in Ion-Solid Interactions: The Initial Stages of Radiation Damage,” {\it Physical Review Letters} {\bf 108}, 213201 (2012).

\end{enumerate}

\subsubsection*{Synergestic Activities}
\begin{itemize}
    \item •	2018–present: QMCPACK Exascale Computing Project (ECP), (PI Paul Kent) provided architectural C++ generic libraries for multi-dimensional arrays interoperable in GPU and CPU memory and generic MPI communication for Real Space and Auxiliar Field QMC.
    \item 2017–present: PI LDRD on linear and non-linear electrical and thermal conductivity from real-time dynamics: the subject closely related to the proposed scientific research subjects. The current core members: Xavier Andrade (LLNL), Alicia Welden (PD, LLNL) and R. Ullah (PD, LLNL).
    \item 2001–present: a user of high-performance ab-initio molecular dynamics codes, GP and Qbox, developed by Francois Gygi of U. C. Davis. This often involved benchmarking on particular high performance computer systems introduced at Lawrence Livermore National Laboratory through the Grand Challenge Supercomputing Program (originally called Science Run under which LUSTRE file system was tested). Outcome of these activities in scientific side led to numerous high-profile publications. He has a history of the first and only implementation of Time-Dependent Density Functional Theory in Qbox (see publication 6 and 9).
    \item 2016–2018: He was Thrust leader at the Energy Frontier Research Center on Energy Dissipation and Defect Evolution (EDDE). As part of that Center, he developed a new theory of electron–phonon coupling scalable to massive number of ions for the simulation of ion cascade and radiation damage (see publication).
\end{itemize}
\clearpage

\subsection*{Xavier Andrade}
Lawrence Livermore National Laboratory: Research Scientist, Quantum Simulation Group, Materials Science Division, Physical and Life Sciences Directorate

\subsubsection*{Education and Training}

\begin{table}[ht]
    \centering
    \begin{tabular}{llll}
        University of Chile & Physics & B.A. & 2002  \\
        University of Chile & Physics & M.S. & 2004 \\
        University of the Basque Country & Physics & PhD & 2010\\
    \end{tabular}
\end{table}

\subsubsection*{Research and Professional Experience}

\begin{table}[ht]
    \centering
    \begin{tabular}{ll}
Staff Scientist, Lawrence Livermore National Laboratory &       2016 - present   \\
Postdoctral Fellow, Lawrence Livermore National Laboratory     &  2014 - 2016   \\
Postdoctral Fellow, Harvard University &       2010 - 2014 \\
    \end{tabular}
\end{table}

\subsubsection*{Publications}

\begin{enumerate}
    \item X. Andrade, S. Hamel, and A. A. Correa, “Negative differential conductivity in liquid aluminum from real-time quantum simulations,” {\it The European Physical Journal B} {\bf 91}, 229 (2018).
    \item E. E. Quashie, B. C. Saha, X. Andrade, and A. A. Correa, “Self-interaction effects on charge-transfer collisions,” {\it Physical Review A} {\bf 95}, 042517 (2017).
    \item E. W. Draeger, X. Andrade, J. A. Gunnels, A. Bhatele, A. Schleife, and A. A. Correa, “Massively parallel first-principles simulation of electron dynamics in materials,” IEEE International Parallel \& Distributed Processing Symposium (2016).
    \item X. Andrade, D. Strubbe, U. De Giovannini, A. H. Larsen, M. J. T. Oliveira, J. Alberdi-Rodriguez, A. Varas, I. Theophilou, N. Helbig, M. Verstreate, L. Stella, F. Nogueira, A. Aspuru-Guzik, A. Castro, M. A. L. Marques, and A. Rubio, “Real-space grids and the Octopus code as tools for the development of new simulation approaches for electronic systems,” {\it Physical Chemistry Chemical Physics} {\bf 17}, 31371-31396 (2015). 
    \item X. Andrade, and A. Aspuru-Guzik, “Real-space density functional theory on graphical processing units: computational approach and comparison to Gaussian basis set methods,” {\it Journal of Chemical Theory and Computation} {\bf 9}, 4360-4373 (2013).
    \item X. Andrade, J. Alberdi-Rodriguez, D. A. Strubbe, M. J. T. Oliveira, F. Nogueira, A. Castro, J. Muguerza, A. Arruabarrena, S. G. Louie, A. Aspuru-Guzik, A. Rubio, and M. A. L. Marques, “Time-dependent
    \item density-functional theory in massively parallel computer architectures: the Octopus project,” {\it Journal of Physics: Condensed Matter} {\bf 24}, 233202 (2012).
    \item X. Andrade, and A. Aspuru-Guzik, “Prediction of the derivative discontinuity in density functional theory from an electrostatic description of the exchange and correlation potential,” {\it Physical Review Letters} {\bf 107}, 183002 (2011).
    \item X. Andrade, A. Castro, D. Zueco, J. L. Alonso, P. Echenique, F. Falceto, and A. Rubio, “A modified Ehrenfest formalism for efficient large scale ab-initio molecular dynamics,” {\it Journal of Chemical Theory and Computation} {\bf 5}, 728–742 (2009).
    \item J. L. Alonso, X. Andrade, P. Echenique, F. Falceto, D. Prada-Gracia, and A. Rubio, “Efficient Formalism for Large-Scale Ab Initio Molecular Dynamics based on Time-Dependent Density Functional Theory,” Physical Review Letters, 101, 96403 (2008).
\end{enumerate}

\subsubsection*{Synergestic Activities}

\begin{itemize}
    \item 2014–present: XA has collaborated with A. A. Correa at LLNL on research program on the application of electron dynamics and time-dependent density functional theory for warm dense matter and other research areas that are fundamental for LLNL’s mission. This has led to secure LDRD and EFRC funding, and several publications (1--3, for example)
    \item 2016–2018: As the main developer in charge of the Qball code, XA collaborated with researchers inside LLNL, including Tuan Anh Pham, Alfredo Correa, and Tadashi Ogitsu, in implementing new and even unique features, and adapting the code for their research work. 
    \item 2005–present: Main developers to the Octopus package. A time-dependent density functional theory software package used by hundreds of users around the world, that contains 220k lines of code. He has made are more than 5,000 commits and contributed 13 million lines of changes to the code. This includes the implementation of the linear response module of Octopus. He also coordinates the development effort of around 20 developers from several countries, to ensure the validity of the programs results, the quality of the code, the usability and the documentation. See publications 4, 8.
    \item 2007–2018: Porting and optimization of electronic structure codes for HPC platforms. Porting of the Octopus code to GPUs (publication 7) obtaining a 5x speed up over the CPU code and up to a 3x speed up over other GPU codes. Massively parallel implementation of Octopus over MPI, scaling up to 400,000 cores (publications 6 and 8), that lead to the calculation of the optical absorption of a photosynthetic complex with 6,000 atoms, the largest TDDFT calculation done up to now (publication 5). Together with A. A. Correa and other LLNL researchers participated in the parallel implementation of TDDFT in the QBall code, that scaled to the full Sequoia supercomputer (1.6 million cores) with near 50\% efficiency (publication 3).
    \item He is regularly invited to present and participate in meetings and workshops of the time-dependent density functional theory community. He contributes to the development of the field (publications 1-2, 9-10) and is up to date in the latest developments in the area.
\end{itemize}
\clearpage

\subsection*{Aaron M. Lindenberg}

Stanford University / SLAC National Accelerator Laboratory: Associate Professor

\subsubsection*{Education and Training}
\begin{table}[ht]
    \centering
    \begin{tabular}{llll}
       Columbia University  & Physics & B.A. & 1996 \\
        University of California, Berkeley & Physics & Ph.D. & 2001 \\
    \end{tabular}
 \end{table}
 
\subsubsection*{Research and Professional Experience}
 
 \begin{table}[ht]
     \centering
     \begin{tabular}{ll}
        Associate Professor, Departments of Materials Science and &  4/1/2015 - Present  \\ 
        Engineering / Photon Science, Stanford University/SLAC &\\ 
        National Accelerator Laboratory & \\
        Assistant Professor, Departments of Materials Science and & 4/1/2007 - 3/31/2015\\
        Engineering and of Photon Science, Stanford University &  \\
        Staff Scientist, Stanford Synchrotron Radiation Laboratory, & 2003 - 2007 \\ 
        Stanford, CA &\\
         Postdoctoral Faculty Fellow, University of California at Berkeley, & 2001 - 2003 \\ 
         Berkeley, CA & \\
     \end{tabular}
 \end{table}
 
\subsubsection*{Publications}
 
\begin{enumerate}
    \item Burak Guzelturk, Antonio Mei, Lei Zhang, Liang Tan, Patrick Donahue, Anisha Singh, Darrell Schlom, Lane Martin, Aaron Lindenberg, “Light-Induced Currents at Domain Walls in Multiferroic BiFeO3,” {\it Nano Lett.} {\bf 20}, 145 (2020). 
    \item I-C. Tung, A. Krishnamoorthy, S. Sadasivam, H. Zhou, Q. Zhang, K.L. Seyler, G. Clark, E.M. Mannebach, C Nyby, F. Ernst, D. Zhu, J.M. Glownia, M.E. Kozina, S. Song, S. Nelson, H. Kumazoe, F. Shimojo, R.K. Kalia, P. Vashishta, P. Darancet, T.F. Heinz, A. Nakano, X. Xu, A. M.Lindenberg, H. Wen, “Anisotropic structural dynamics of monolayer crystals revealed by femtosecond surface x-ray scattering,” {\it Nat. Photon.} {\bf 13}, 425 (2019).
    \item E. Y. Ma, B. Guzelturk, G. Li, L. Cao, Z.-X. Shen, A. M. Lindenberg, T. F. Heinz, “Recording interfacial currents on the sub-nanometer length and femtosecond time scale by terahertz emission,” {\it Science Advances} 5:eaaau0073 (2019).
    \item Edbert J. Sie, Clara M. Nyby, C. D. Pemmaraju, Su Ji Park, Xiaozhe Shen, Jie Yang, Matthias C. Hoffmann, B. K. Ofori-Okai, Renkai Li, Alexander H. Reid, Stephen Weathersby, Ehren Mannebach, Nathan Finney, Daniel Rhodes, Daniel Chenet, Abhinandan Antony, Luis Balicas, James Hone, Thomas P. Devereaux, Tony F. Heinz, Xijie Wang, Aaron M. Lindenberg, “An ultrafast symmetry switch in a Weyl semimetal,” {\it Nature} {\bf 565}, 61 (2019).
    \item Burak Guzelturk, Rebecca A. Belisle, Matthew D. Smith, Karsten Bruening, Rohit Prasanna, Yakun Yuan, Venkatraman Gopalan, Christopher J. Tassone, Hemamala I. Karunadasa, Michael D. McGehee, Aaron M. Lindenberg, “Terahertz Emission from Hybrid Perovskites Driven by Ultrafast Charge Separation and Strong Electron-Phonon Coupling,” {\it Advanced Materials} {\bf 30}, 1704737 (2018). 
    \item Y. Qi, S. Liu, A. M. Lindenberg, and A. M. Rappe, “Ultrafast electric field pulse control of giant temperature change in ferroelectrics,” {\it Phys. Rev. Lett.} {\bf 120}, 055901 (2018).
    \item E. M. Mannebach, C. Nyby, F. Ernst, Y. Zhou, J. Tolsma, Y. Li, M. Sher, I-C Tung, H. Zhou, Q. Zhang, K.L. Seyler, G. Clark, Y. Lin, D. Zhu, J.M. Glownia, M.E. Kozina, S. Song, S. Nelson, A. Mehta, Y. Yu, A. Pant, O. Aslan, A. Raja, Y. Guo, A. DiChiara, W. Mao, L. Cao, S. Tongay, J. Sun, D.J. Singh, T.F. Heinz, X. Xu, A.H. MacDonald, E. Reed, H. Wen, A.M. Lindenberg, “Dynamic optical tuning of interlayer interactions in the transition metal dichalcogenides,” {\it Nano Lett.} {\bf 17}, 7761 (2017).
    \item X. Wu, L. Z. Tan, X. Shen, T. Hu, K. Miyata, M. Tuan Trinh, R. Li, R. Coffee, S. Liu, D.A. Egger, I. Makasyuk, Q. Zheng, A. Fry, J.S.Robinson, M.D. Smith, B. Guzelturk, H.I. Karunadasa, X. Wang, X.-Y. Zhu, L. Kronik, A.M. Rappe, A.M. Lindenberg, “Light-induced picosecond rotational disordering of the inorganic sublattice in hybrid perovskites,” {\it Science Advances,} 3:e1602388 (2017).  
    \item A.M. Lindenberg, S.L. Johnson, D.A. Reis, “Visualization of atomic-scale motions in materials via femtosecond x-ray scattering techniques,” {\it Annual Review of Materials Research} {\bf 47}, 15 (2017).
    \item Peter Zalden, Michael J. Shu, Frank Chen, Xiaoxi Wu, Yi Zhu, Haidan Wen, Scott Johnston, Zhi-Xun Shen, Patrick Landreman, Mark Brongersma, Scott W. Fong, H.-S.Philip Wong, Meng-Ju Sher, Peter Jost, Matthias Kaes, Martin Salinga, Alexander von Hoegen, Matthias Wuttig, and Aaron Lindenberg, “Picosecond electric-field-induced threshold switching in phase-change materials,” {\it Phys. Rev. Lett.} {\bf 117}, 067601 (2016). 
\end{enumerate}

\subsubsection*{Synergestic Activities}

\begin{itemize}
    \item Program Committee, Conference on Ultrafast Phenomena (2018)
    \item International Program Committee, SPIE conference on Synthesis and Photonics of Nanoscale Materials XVI
    \item Advisory Board, International Symposium on Ultrafast Phenomena and Terahertz Waves (2017–2018)
    \item Guest Editor, MRS Bulletin Special Issue on Ultrafast Imaging (2017–2018)
    \item Co-organizer, APS March Meeting Focus Topic on Van der Waals bonding in advanced materials (2017–2018)
\end{itemize}

\clearpage

\subsection*{Sri Chaitanaya Das Pemmeraju}

SLAC National Accelerator Laboratory Associate Staff Scientist, Stanford Institute for Materials and Energy Sciences

\subsubsection*{Education and Training}

\begin{table}[ht]
    \centering
    \begin{tabular}{llll}
       Andhra University, India  & Mathematics, Physics & B. Sc & 2001 \\
       & \& Computer Science & & \\
        Indian Institute of Technology, & & &\\
        Madras, India & Physics & M. Sc & 2003 \\
        Trinity College Dublin, Ireland & Physics & Ph.D & 2008 \\
    \end{tabular}
\end{table}

\subsubsection*{Research and Professional Experience}

\begin{table}[ht]
    \centering
    \begin{tabular}{ll}
       Associate Staff Scientist, Stanford Institute for Materials and & 2017 - Present   \\
      Energy Sciences, SLAC National Accelerator Laboratory & \\
      Project scientist in the Materials Sciences Division, & 2013 - 2017 \\
      Lawrence Berkeley National Laboratory & \\
      Postdoctoral fellow, & 2012 - 2013 \\
      Chemical Sciences Division, Lawrence Berkeley National Laboratory & \\
        Postdoctoral fellow in Computational Spintronics, & 2008 - 2012 \\
        Trinity College Dublin, Ireland & \\
    \end{tabular}
\end{table}

\subsubsection*{Publications}

\begin{enumerate}
    \item “Simulation of attosecond transient soft X-ray absorption in solids using generalized Kohn Sham real-time TDDFT”, C. D. Pemmaraju, {\it New J. Phys.} {\bf 22}, 083063 (2020). https://doi.org/10.1088/1367-2630/aba76c
    \item “Valence and core excitons in solids from velocity-gauge real-time TDDFT with range-separated hybrid functionals: An LCAO approach,” C.D. Pemmaraju, {\it Computational Condensed Matter} {\bf 18}, e00348 (2019).  https://doi.org/10.1016/j.cocom.2018.e00348
    \item “Velocity-gauge real-time TDDFT within a numerical atomic orbital basis set,” C.D. Pemmaraju, F.D. Vila, J.J. Kas, S.A. Sato, J.J. Rehr, K. Yabana, David Prendergast, {\it Computer Physics Communications} 2018, ISSN 0010-4655, https://doi.org/10.1016/j.cpc.2018.01.013.
    \item “Efficient implementation of core-excitation Bethe–Salpeter equation calculations.” K. Gilmore,  John Vinson, E.L. Shirley, D. Prendergast, C. D. Pemmaraju,  J.J. Kas, F.D. Vila,  J.J. Rehr, {\it Computer Physics Communications}, http://dx.doi.org/10.1016/j.cpc.2015.08.014.
    \item “Attosecond band-gap dynamics in silicon.” Martin Schultze, Krupa Ramasesha, C.D. Pemmaraju, S.A. Sato, D. Whitmore, A. Gandman, James S. Prell, L. J. Borja, D. Prendergast, K. Yabana, Daniel M. Neumark, and Stephen R. Leone. {\it Science} {\bf 346}, 1348 (2014).
    \item “Femtosecond x-ray spectroscopy of an electrocyclic ring-opening reaction.” Andrew R. Attar, Aditi Bhattacherjee, C. D. Pemmaraju, Kirsten Schnorr, Kristina D. Closser, David Prendergast, Stephen R. Leone, {\it Science} {\bf 356}, 54–59 (2017).
    \item “Soft X-Ray Second Harmonic Generation as an Interfacial Probe.” Royce K Lam, SL Raj, TA Pascal, CD Pemmaraju, L Foglia, A Simoncig, N Fabris, P Miotti, CJ Hull, AM Rizzuto, JW Smith, R Mincigrucci, C Masciovecchio, A Gessini, E Allaria, G De Ninno, B Diviacco, E Roussel, S Spampinati, G Penco, S Di Mitri, M Trovò, M Danailov, ST Christensen, D Sokaras, T-C Weng, M Coreno, L Poletto, WS Drisdell, D Prendergast, L Giannessi, E Principi, D Nordlund, RJ Saykally, CP Schwartz, {\it Phys. Rev. Lett.} {\bf 120}, 023901 (2018).
    \item “Direct and Simultaneous Observation of Ultrafast Electron and Hole Dynamics in Germanium.” Zürch, M., Chang, H.T., Borja, L.J., Kraus, P.M., Cushing, S.K., Gandman, A., Kaplan, C.J., Oh, M.H., Prell, J.S., Prendergast, D., Pemmaraju, C. D., Neumark D. M., and Leone S. R. {\it Nat. Comm.} {\bf 8}, 15734 (2017).
    \item “Direct observation of ring-opening dynamics in strong-field ionized selenophene using femtosecond inner-shell absorption spectroscopy,” Florian Lackner, Adam S. Chatterley, C. D. Pemmaraju, Kristina D. Closser, David Prendergast,  Daniel M. Neumark, Stephen R. Leone, and  Oliver Gessner, {\it The Journal of Chemical Physics} {\bf 145}, 234313 (2016).
    \item “Detecting the oxyl radical of photocatalytic water oxidation at an n-SrTiO3/aqueous interface through its subsurface vibration,” David M. Herlihy, Matthias M. Waegele, Xihan Chen, C. D. Pemmaraju, David Prendergast and Tanja Cuk, {\it Nature Chemistry} {\bf 8}, 549–555 (2016).
\end{enumerate}

\subsubsection*{Synergestic Activities}

\begin{itemize}
    \item Organized a number of workshops on theoretical simulation of spectroscopies at SLAC/Stanford.
    \item Mentored several batches of  Stanford, CCI and STEMCore summer interns at SLAC from 2017-2020 
    \item Member of the American Physical Society since 2006
    \item Reviewer for Physical Review Letters, Physical Review B, Nature Scientific Reports, Euro Physics Letters, Chemical Physics Letters, ACS Omega, Surface Science and others 

\end{itemize}

\clearpage

\subsection*{Liang Z. Tan}

Lawrence Berkeley National Laboratory
Staff Scientist

\subsubsection*{Education and Training}

\begin{table}[ht]
    \centering
    \begin{tabular}{llll}
        California Institute of Technology & Physics & B.S. & 2008 \\
        University of California at Berkeley & Physics & Ph.D. & 2014 \\
    \end{tabular}
\end{table}

\subsubsection*{Research and Professional Experience}

\begin{table}[ht]
    \centering
    \begin{tabular}{ll}
        Staff Scientist, The Molecular Foundry,  & 2018 - Present   \\
        Lawrence Berkeley National Laboratory & \\
        Postdoctral Fellow, University of Pennsylvania & 2014 - 2018 \\
    \end{tabular}
\end{table}

\subsubsection*{Publications}
\begin{enumerate}
    \item J Hong, D Prendergast, and LZ Tan, “Layer Edge States Stabilized by Internal Electric Fields in Two-Dimensional Hybrid Perovskites”, {\it Nano Lett.} {\bf 21}, 182–188 (2021).
    \item Sangeeta Rajpurohit, LZ Tan, Christian Jooss, and P. E. Blöchl, “Ultrafast spin-nematic and ferroelectric phase transitions induced by femtosecond light pulses”, {\it Phys. Rev. B} {\bf 102}, 174430 (2020).
    \item B Guzelturk, AB Mei, L Zhang, LZ Tan, P Donahue, AG Singh, DG Schlom, LW Martin, AM Lindenberg, Light-Induced Currents at Domain Walls in Multiferroic BiFeO3, {\it Nano Lett.} {\bf 20}, 145 (2020).
    \item MZ Mayers, LZ Tan, DA Egger, AM Rappe, DR Reichman, “How Lattice and Charge Fluctuations Control Carrier Dynamics in Halide Perovskites,” {\it Nano Lett.} {\bf 18}, 8041-8046 (2018). DOI: 10.1021/acs.nanolett.8b04276 
    \item JPH Rivett, LZ Tan, MB Price, SA Bourelle, NJLK Davis, J Xiao, Y Zou, R Middleton, B Sun, AM Rappe, D Credgington and F Deschler, “Long-lived polarization memory in the electronic states of lead-halide perovskites from local structural dynamics,” {\it Nat. Comm.} {\bf 9}, 3531 (2018).
    \item X Wu, LZ Tan, X Shen, T Hu, K Miyata, MT Trinh, R Li, R Coffee, S Liu, DA Egger, I Makasyuk, Q Zheng, A Fry, JS Robinson, MD Smith, B Guzelturk, HI Karunadasa, X Wang, XY Zhu, L Kronik, AM Rappe, AM Lindenberg, “Light-induced picosecond rotational disordering of the inorganic sublattice in hybrid perovskites,” {\it Science Advances} {\bf 3}, e1602388 (2017) DOI: 10.1126/sciadv.1602388
    \item Yaffe, Y Guo, LZ Tan, DA Egger, T Hull, C Stoumpos, F Zheng, TF Heinz, L Kronik, MG Kanatzidis, JS Owen, AM Rappe, MA Pimenta, LE Brus, “Local Polar Fluctuations in Lead Halide Perovskite Crystals,” {\it Phys. Rev. Lett.} {\bf 118}, 136001 (2017) DOI:https://doi.org/10.1103/PhysRevLett.118.136001
    \item LZ Tan, F Zheng, SM Young, F Wang, S Liu, AM Rappe, “Shift Current Bulk Photovoltaic Effect in Polar Materials - hybrid and oxide perovskites and beyond,” {\it NPJ Comput. Mat.} {\bf 2}, 16026 (2016) DOI: https://doi.org/10.1038/npjcompumats.2016.26
    \item LZ Tan, AM Rappe, “Enhancement of Bulk Photovoltaic Effect in Topological Insulators,” {\it Phys. Rev. Lett.} {\bf 116}, 237402 (2016) DOI:https://doi.org/10.1103/PhysRevLett.116.237402
    \item F Zheng, LZ Tan, S Liu, AM Rappe, “Rashba Spin-Orbit Coupling Enhanced Carrier Lifetime in CH3NH3PbI3,” {\it Nano Lett.} 15 (12), 7794-7800 (2015) DOI: 10.1021/acs.nanolett.5b01854
\end{enumerate}

\subsubsection*{Synergestic Activities}
\begin{itemize}
    \item Liang Tan’s recent invited lectures were at the University of Oxford Department of Materials Science, at the University of California, Davis Department of Physics Colloquium, at the SPIE Nanoscience and Engineering conference, and at the Computational Simulation Studies workshop at the University of Georgia. 
    \item He has organized symposia at the Molecular Foundry Annual User Meeting (2021) and at the American Chemical Society (Fall 2021), on the topic of non-equilibrium science. He is the organizer of the quantum materials theory seminar at the Molecular Foundry at LBL. 
    \item As part of STEM young researcher mentoring, he has served as a mentor to undergraduate researchers at the University of California, Berkeley, under the BLUR program. 
    \item Liang Tan has served as a reviewer for Nature Communications, Nanoscale, Physical Review B, Journal of Applied Physics, and Physical Chemistry Chemical Physics.  
\end{itemize}

\clearpage

\subsection*{David G. Prendergast}

Lawrence Berkeley National Laboratory: Facility Director, Theory of Nanostructured Materials, The Molecular Foundry

\subsubsection*{Education and Training}

\begin{table}[ht]
    \centering
    \begin{tabular}{llll}
        University College Cork & Physics \& Mathematics & BSc & 1999 \\
        University College Cork & Physics & PhD & 2002 \\
    \end{tabular}
\end{table}

\subsubsection*{Research and Professional Experience}

\begin{table}[ht]
    \centering
    \begin{tabular}{ll}
        Facility Director, Theory of Nanostructured Materials, & 2020 - present \\
        Interim Division Director, The Molecular Foundry, LBNL & 2019-2020 \\ 
    Senior Staff Scientist, Molecular Foundry, LBNL & 2017 - present    \\
        Facility Director, Theory of Nanostructured Materials, & 2012 - 2019 \\
        Molecular Foundry, LBNL & \\
        Staff Scientist, Molecular Foundry, LBNL (Group Lead: Jeff Neaton) & 2007 - 2017 \\
        Postdoctral Fellow, University of California, Berkeley & 2005 - 2007 \\
        Postdoctral Fellow, Lawrence Livermore National Laboratory & 2002 - 2005 \\
    \end{tabular}
\end{table}

\subsubsection*{Publications}
\begin{enumerate}
    \item Yierpan Aierken, Ankit Agrawal, Meiling Sun, Marko Melander, Ethan J. Crumlin, Brett A. Helms, and David Prendergast, “Revealing Charge-Transfer Dynamics at Electrified Sulfur Cathodes Using Constrained Density Functional Theory,” {\it J. Phys. Chem. Lett.} {\bf 12}, 739 (2021).
    \item Han Wang, Michael Odelius, David Prendergast, “A combined multi-reference pump-probe simulation method with application to XUV signatures of ultrafast methyl iodide photodissociation,” {\it J. Chem. Phys.} {\bf 151}, 124106 (2019).
    \item Yufeng Liang, John Vinson, S. C. Pemmaraju, Walter S. Drisdell, Eric L. Shirley, David Prendergast, “Accurate X-Ray Spectral Predictions: An Advanced Self-Consistent-Field Approach Inspired by Many-Body Perturbation Theory,” {\it Phys. Rev. Lett.} {\bf 118}, 096402 (2017).
    \item Taylor A. Barnes, Thorsten Kurth, Pierre Carrier, Nathan Wichmann, David Prendergast, Paul R. C. Kent, Jack Deslippe, “Improved treatment of exact exchange in Quantum ESPRESSO,” {\it Comp. Phys. Comm.} {\bf 214}, 52 (2017).
    \item C.D. Pemmaraju, F.D. Vila, J.J. Kas, S.A. Sato, J.J. Rehr, K. Yabana, D. Prendergast, “Velocity-gauge real-time TDDFT within a numerical atomic orbital basis set,” {\it Computer Physics Communications} {\bf 226}, 30 (2018).
    \item A.R. Attar, A. Bhattacherjee, C.D. Pemmaraju, K. Schnorr, K.D. Closser, D. Prendergast, S.R. Leone, “Femtosecond x-ray spectroscopy of an electrocyclic ring-opening reaction,” {\it Science} {\bf 356}, 54 (2017).
    \item M. Zuerch, H.-T. Chang, L.J. Borja, P.M. Kraus, S.K. Cushing, A. Gandman, C.J. Kaplan, M.H. Oh, J.S. Prell, D. Prendergast, C.D. Pemmaraju, D.M. Neumark, S.R. Leone, “Direct and simultaneous observation of ultrafast electron and hole dynamics in germanium” {\it Nature Communications} {\bf 8}, 15734 (2017).
    \item X. Chen, S.N. Choing, D.J. Aschaffenburg, C.D. Pemmaraju, D. Prendergast, and T. Cuk, “The Formation Time of Ti–O• and Ti–O•–Ti Radicals at the n-SrTiO3/Aqueous Interface during Photocatalytic Water Oxidation,” {\it J. Am. Chem. Soc.} {\bf 139}, 1830 (2017).
    \item F. Lackner, A.S. Chatterley, C. D. Pemmaraju, K.D. Closser, D. Prendergast, D.M. Neumark, S.R. Leone, and O. Gessner, “Direct observation of ring-opening dynamics in strong-field ionized selenophene using femtosecond inner-shell absorption spectroscopy,” {\it J. Chem. Phys.} {\bf 145}, 234313 (2016). 
    \item B. Cho, T. Ogitsu, K. Engelhorn, A. Correa, Y. Ping, J. Lee, L. Bae, D. Prendergast, R. Falcone, P. Heimann, “Measurement of Electron-Ion Relaxation in Warm Dense Copper,” {\it Sci. Rep.} {\bf 6}, 18843 (2016).
\end{enumerate}

\subsubsection*{Synergestic Activities}
\begin{itemize}
    \item Reviewer for Phys Rev Lett, Phys Rev B, J Chem Phys, J Phys Chem C, J Phys Chem B, J Phys Chem Lett, Chem Mat, J El Spectr Rel Phenom, Nat Comms, J Chem Theo Comp
    \item Advanced Light Source Scientific Advisory Council Member
    \item Panel Co-Chair for 2017 DOE-BES Basic Research Needs Workshop on Advanced Electrical Energy Storage
    \item Member of the VUVX International Scientific Committee (2013 - present)
    \item Molecular Foundry Division Staff Committee Chair
\end{itemize}
\clearpage