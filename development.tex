\section{Development And Utilization Of Unique Facilities, Capabilities, Or Approaches}
\label{sec:development}
{\small\color{red}
\begin{itemize}
\item 3 pages max
    \item Describe the most important instrumentation, computer codes, and databases purchased or developed for the project and their pertinent capabilities. Provide 
    \begin{enumerate}
        \item software release numbers,
        \item user metric where available,
        
        \item citations of the accompanying publication of new software releases.
        \item Links to software and database download sites should be provided in separate spreadsheet (see below).
        \item Also describe any unique shared facilities (including national user facilities) that have been or will be used for the project, including ALCC or INCITE awards.
    \end{enumerate}    
\end{itemize}

}

Talk about libraries used by QMCPACK also.

\begin{figure}
    \centering
    \includegraphics{figures/placeholder.jpg}%{development/lines_overlay.pdf}
    \caption{
        Inq source-code lines \emph{vs.} time since the start of the project.  
        Note that test and code are developed side-by-side, and the amount of lines of code is comparable.  We     find testing fundamental for continuous development and maintenance of the code. 
        The dip near 11/2019 reflects the removal of internal exchange and corrections (XC) functional computation, which was replaced by the external library libxc. 
        The relatively low line count (12,000) for a fully functional DFT code is achieved by delegating functionality to libxc (shown), Multi and B.MPI3 libraries.}
    \label{fig:lines_overlay}
\end{figure}

\clearpage