\section{Development And Utilization Of Unique Facilities, Capabilities, Or Approaches}
\label{sec:development}
{\small\color{red}
\begin{itemize}
\item 3 pages max
    \item Describe the most important instrumentation, computer codes, and databases purchased or developed for the project and their pertinent capabilities. Provide 
    \begin{enumerate}
        \item software release numbers,
        \item user metric where available,
        
        \item citations of the accompanying publication of new software releases.
        \item Links to software and database download sites should be provided in separate spreadsheet (see below).
        \item Also describe any unique shared facilities (including national user facilities) that have been or will be used for the project, including ALCC or INCITE awards.
    \end{enumerate}    
\end{itemize}

}

Grand Challenge 

Outreach summer school director, two students.

QMCPACK library

As requested by the program, one of the central objectives of the first half of the NPNEQ center was the delivery of a performant GPU DFT code to tackle non-perturbative and non-equilibrium phenomena studied by the center.

Talk about libraries used by QMCPACK also. interaction with other projects


Outreach activities include the organization of a symposium at the 2021 Molecular Foundry Annual User meeting (August 19-20, 2021). 
The symposium, entitled ``Non-Equilibrium and Non-Perturbative Experimental and Theoretical Methods for Dynamical Science'', aims to survey recent progress in manipulation, measurement, and simulation of non-equilibrium dynamics.
Alongside the symposium organization, the \textsc{inq} team will also be presenting a tutorial-style introduction to the code during the Molecular Foundry User Meeting, as a way to introduce the code to the electronic structure community.

\clearpage

