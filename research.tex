\section{Description Of Research Tasks, Progress, And Plans}
\label{sec:research}

{\small\color{red}
\begin{itemize}
\item 12 pages max
    \item Include an Executive Summary of the progress made since September 2019 toward
meeting its strategic objectives, making clear how the accomplishments described in detail in Section 6 below fit together as part of a synergistic research program.
\begin{enumerate}
    \item Specific research objectives associated with each task
    \item Research progress for each task since the start of the project
    \item Integration of research activities across tasks
    \item Research planned for the balance of the award period (through September 2023).
\end{enumerate}
\item In the descriptions of the tasks and their integration, the need for a collaborative, synergistic approach involving several investigators should be clearly established. The role and intellectual contribution of each senior investigator should be clear to the reviewer, and the availability of the resources necessary to accomplish the research objectives should be evident. Approximately 70\% of this section should be dedicated to research progress to date and 30\% to planned research.
    \item \textcolor{red}{Review criteria for Scientific and/or Technical Merit of the Proposed Research:}
    \begin{itemize}
        \item \textcolor{red}{What new capability/functionality will the proposed software provide to the materials research community? How will the research plan attain the 4-year research and software/data goals?}
        \item \textcolor{red}{Comment on the novelty and scientific value of the proposed research. }
        \item \textcolor{red}{How widespread would be the interest in the software within the materials research community?}
        \item \textcolor{red}{How does the proposed work compare with other efforts in its field, both in terms of scientific and/or technical merit and originality?}
        \item \textcolor{red}{Comment on the progress and impact for the current award period.}
    \end{itemize}

\end{itemize}
}
{\color{green}
Note to ourselves:
\begin{itemize}
    \item TB-INQ integration in future: forced oscillation? any idea? Keep in mind: non perturbative aspect is very important.
    \item k-points and non-orthgonal cell
    \item User defined operation for external collaborator? How to balance with performance?
    \item How to expand user base need to be spelled out: Online tutorial by Xavier in late summer, ESW, TMF users meeting (Liang gave us material), SLAC (name?) workshop, US-Africa collaboration (?)
    \item CCMS to expand collaboration: two summer student from Andre and ? with Alfredo and Kejun from Yuan with Tadashi. Perhaps, get theirs support information and relevant research area to make it more convincing and compelling.
\end{itemize}

}
\clearpage

\section{Progress and Accomplishments to date}

As outlined above, the primary strategic objective of the NPNEQ CMS center during the first 18 months of the funding period was the development of a \textbf{portable}, \textbf{extensible}, \textbf{scalable} and \textbf{reliable} DFT/RT-TDDFT framework that is fully compatible with existing petascale and near-term exascale \textbf{hybrid CPU-GPU} architectures.
This objective was achieved with the recent release of \textsc{inq}~\cite{Andrade2021}. 
In parallel however, the team also pursued multi-institutional collaborative research efforts leading to science outcomes in the area of \emph{ultrafast materials science} using alternate tools and methodologies, some of which will be incorporated into the larger INQ ecosystem of codes in future. 
In the following we provide a summary of both the software development and scientific research efforts within NPNEQ.

\subsection{Code Development}

\subsubsection{The INQ DFT/TDDFT platform} 

With the widespread adoption of Graphical Processing Units (GPUs) as a powerful and energy efficient hardware paradigm for scientific computing over the past fifteen years and with the ongoing migration of DOE's national super-computing facilities from CPU to networked hybrid CPU-GPU architectures, it is of paramount importance within computational materials science that the community has access to electronic structure codes that are fully compatible with the emerging GPU computing landscape.
The INQ platform~\cite{Andrade2021} under development by the NPNEQ team addresses this need within the context of Density Functional Theory (DFT) based approaches to materials simulation while further specializing to nonperturbative simulations of materials under nonequilibrum conditions.
During the first 18 months of this funding period, the NPNEQ center accomplished the design and development tasks necessary to deliver a modular and extensible GPU-accelerated software solution for DFT/TDDFT led by LLNL PIs \textbf{Andrade} and \textbf{Correa}.
In the following we outline the unique design features of INQ as well as it's current capabilities:

\paragraph{General features and design:}
The design philosophy of INQ differs in many important respects from established Fortran/C based DFT/TDDFT codes developed in previous decades to run on CPUs: Unlike the CPU computing model that matured decades ago, GPU computing is still rapidly evolving with different hardware designs competing for primacy in the super-computing space and vendors making significant low-level updates and optimizations on a regular basis.
\textsc{Inq} is therefore designed from the outset to be easily modified and improved on top of evolving hardware, while providing consistent results and behavior independent of specific vendor platforms. 
\textsc{Inq} achieves this goal through abstractions enabled by modern C++ programming, by implementing a clear separation between different classes of software components, namely: application (DFT/TDDFT) specific code, infrastructure (basis-set, boundary conditions, data-structures etc) code and low-level numerical library (blas, FFTs etc) code. 
\newline 
\newline
As shown in Fig.~\ref{fig:inq_design}(a), in traditional DFT/TDDFT programs, the different software components are rather tightly coupled into monolithic software units and the programmer may encounter all classes of code within a given subroutine or function. 
\textsc{Inq} on the other hand implements a hierarchical structure with an application (DFT/TDDFT) code layer that sits on top of an infrastructure library layer which gracefully handles all hardware-dependent decisions. 
A programmer working with the DFT/TDDFT layer then encounters a much simplified programming interface that syntactically resembles the relevant physics equations~(see Ref.~\cite{Andrade2021}) and is shielded from lower level representation-specific and hardware-specific details. 
This hierarchical design is critical for developing code that is hardware portable while simultaneously reducing the learning curve for new developers from the DFT community to be able to quickly add functionality in the application layer. 
Note that while interpreted scripting languages such as Python, MATLAB, etc. can provide a similar type of abstraction, there is a performance cost associated with them which is not incurred by \textsc{inq} as it relies on compiled C++.

\begin{figure}[h]
	\centering
	\includegraphics[width=1.0\linewidth]{figures/INQ_design_features.pdf}
	\caption{
		Key design features of the INQ framework.
		(a) INQ was built from the outset as a library that can operate on both CPU and GPU hardware,
		(b) At roughly 12,000 lines of code, INQ is extremely lightweight compared to other DFT/TDDFT codes offering comparable functionality.
		Adopting GPU-compatible \textsc{libxc} substantially reduced the size of INQ's internal codebase.
		(c) INQ brings together a highly modular hierarchy of libraries ensuring maximum adaptability with respect to future hardware and software changes.
	}
	\label{fig:inq_design}
\end{figure}

The hardware portability of \textsc{inq} is achieved in practice through a highly modular design of its library layer which brings together a number of different components that were developed both at LLNL and by the extended DFT community. 

Some of the key libraries used in INQ are:
\begin{itemize}
	\item \textbf{MULTI:} Developed by PI \textbf{Correa} at LLNL, \textsc{MULTI} provides multidimensional array access on both CPU and GPU memory models while abstracting operations over arrays without compromising maximum performance. 
	It also provides the interface between \textsc{inq} and vendor provided linear algebra, FFT and C++ standard libraries.
	\item \textbf{B.MPI3:} B-MPI3 is a C++ library wrapper for version 3.1 of the Message Passing Interface (MPI) standard that simplifies the its utilization while maintaining a similar level of performance as the more cumbersome native MPI C-language interface. 
	B.MPI3 was developed by PI \textbf{Correa} at LLNL.
	\item \textbf{PSEUDOPOD:} \textsc{PSEUDOPOD} is a CPU+GPU compatible library that handles pseudopotential (PP) parsing, filtering and other auxiliary tasks while supporting many popular PP formats as well standardized PP sets such as Pseudodojo and SG15. 
	\textsc{PSEUDOPOD} developed by PI \textbf{Andrade} at LLNL, moves PP handling from the application layer to the library layer of INQ.
	\item \textbf{LIBXC:} Libxc is a standalone library of exchange and correlation (XC) functionals initially developed by M. Marques. 
	As part of the development on \textsc{inq}, PI \textbf{Andrade} implemented a GPU extension of \textsc{libxc} that relies on cuda ‘unified memory’ to store its internal data structures. 
	\textsc{INQ} therefore has the ability to evaluate XC functionals on GPUs, which is critical to maintain all data in the GPU.
\end{itemize}

Figure~\ref{fig:inq_design}(c) shows the interaction between these \textsc{inq} modules and lower-level vendor provided libraries. 
The modularization of \textsc{inq}'s library layer affords several advantages: For instance, different components can be developed and tested independently as well as shared with other codes, to avoid duplication of work and enable collaborative development. 
Following changes to low-level hardware-optimized libraries, only a small portion of the code needs to be updated in a way that is transparent to application layer developers. 
Furthermore, by creating libraries and offloading tasks to them whenever possible, the codebase internal to \textsc{inq} can be kept relatively small as shown in Fig.~\ref{fig:inq_design}(b). 

Finally, in order to ensure reliability within a collaborative environment, \textsc{inq} adopts a test-driven development (TDD) approach in line with modern software development best practices. 
Therefore the \textsc{inq} codebase consists of both unit and functional tests that are run regularly within an automated continuous delivery platform hosted on gitlab. This workflow ensures that only code modifications that do not lead to errors anywhere in the codebase are accepted.

The result is a fully functional yet light-weight DFT/TDDFT framework that is portable across hardware platforms, offers optimal scaling on distributed hybrid CPU-GPU platforms combined with reliable numerical accuracy. 
\textsc{Inq}'s DFT/TDDFT application layer currently implements semi-local DFT total energies, forces and real-time propagation of the time-dependent KS equations within the adiabatic XC approximations for both molecular and solid-state systems within a periodic supercell framework. 
In the following, we discuss performance and validation tests of these capabilities within \textsc{inq}. 

\paragraph{GPU-MPI functionality and scaling:}
1-1.5 pages including figures

\textsc{Inq} is the first implementation of density functional theory (DFT) and time-dependent DFT (TDDFT) that is written \emph{from scratch} to work on graphical processing units (GPUs).
It proven quite challenging to adapt an existing code to run on the GPU, as extensive modifications to the whole code are needed.
For this reason we decided to start a new code, designed from the ground up to run on GPUs (but that also runs on CPUs). 

We studied the different available platforms for high level GPU programming like \textsc{raja}~\cite{Beckingsale2019}, \textsc{Kokkos}~\cite{CarterEdwards2014}, \textsc{OpenMP}~\cite{Lee2010} and \textsc{SyCL}~\cite{Alpay2020}.
Unfortunately we could not find one that was directly suitable for the operations needed for DFT/TDDFT and that was mature enough at the moment we started the project (mid-2019).
So we decided to design our code using the \textsc{cuda} C++ extensions with a thin layer of abstraction on top, that allows us to make most of the code independent of the GPU backend.

\textsc{Inq} achieves distributed memory parallelism (parallelism between several nodes) by using the Message Passing Interface (\textsc{MPI}) infrastructure.
For simplicity, it uses one MPI task per GPU, for example in LLNL's lassen each of the 4 GPU in a node is controlled by 1 MPI task.

In \textsc{inq} we directly call \textsc{MPI} over data in GPU memory.
Ideally, most modern \textsc{MPI} implementations are GPU-aware and can recognize the GPU memory space and they can directly access data.
In the best scenario they can also communicate the data to and from GPU memory without passing through main memory, but this is not always the case.
For non-GPU-aware implementations, \emph{managed memory} residing on the GPU would be automatically copied to main memory before \textsc{MPI} calls.
Unfortunately this last case has a non-negligible overhead in the communication cost.

\begin{figure}[h]
	\centering
	\includegraphics[width=1.0\linewidth]{figures/scaling/strong}
	\caption{
		Strong scaling plot: wall time per simulation step.
		These simulations are run in LLNL's Lassen supercomputer (4 NVIDIA Tesla P100 Pascal GPUs per compute node).
	}
	\label{fig:scaling_strong}
\end{figure}

\begin{figure}[h]
	\centering
	\includegraphics[width=1.0\linewidth]{figures/scaling/gpu_vs_cpu}
	\caption{
		Comparison of INQ GPU and CPU implementations.
		These simulations are run in LLNL's Lassen supercomputer (4 NVIDIA Tesla P100 Pascal GPUs per compute node).
	}
	\label{fig:scaling_gpu_vs_cpu}
\end{figure}

\begin{figure}[h]
	\centering
	\includegraphics[width=1.0\linewidth]{figures/scaling/weak}
	\caption{
		Best timing achieved for variable system size (\emph{Weak} scaling).
		These simulations are run in LLNL's Lassen supercomputer (4 NVIDIA Tesla P100 Pascal GPUs per compute node).
	}
	\label{fig:scaling_weak}
\end{figure}

The main limitation to scalability is the lack of a parallel FFT library for GPUs. 
Our own basic implementation doesn't scale very well. 
This mean our parallelization only covers one degree of freedom (the states) and not the space (domains). 
That's why we see the time per iteration in the weak scaling going up in Fig.~\ref{fig:scaling_weak}.

Despite the incomplete parallelization, iterator times are below a second per step even with a few hundred atoms, which is very competitive among alternative codes.
Overall, we developed a code that scales extremely well in modern clusters with GPUs.

\paragraph{Validation of results:}
\begin{figure}[h]
	\centering
	\includegraphics[width=1.0\linewidth]{figures/Results-Fig.pdf}
	\caption{
		(Adapted from Ref.~\cite{Andrade2021}.) 
		Comparison between \textsc{inq} and the established electronic structure codes \textsc{Quantum Espresso} (QE) and \textsc{octopus}.
Results for ground state total energies and real-time TDDFT optical response are shown. 
		(a) Total energy vs bond length in \(\mathrm{N_2}\).
		(b) Total energy vs lattice constant in bulk Silicon. 
		(c) Time evolution of the dipole moment in gas phase \(\mathrm{H_2O}\) following a ``kick'' perturbation. 
		(d) Linear optical absorption of molecular \(H_2O\). 
		In all cases there is a high-level of agreement between the codes.
	}
	\label{fig:inq_results}
\end{figure}

For a new scientific code it is fundamental to produce results that are consistent with other approaches and codes. 
As \textsc{inq} follows a test-driven development approach, \textsc{inq}'s codebase is continuously tested and validated at multiple levels against analytical results or benchmark data obtained from other established codes such as \textsc{Quantum Espresso}. 
We recently carried out higher level end-to-end tests of \textsc{inq} as a part of the software release validating \textsc{inq} results against corresponding quantities obtained from established plane-wave and realspace grid codes: \textsc{Quantum Espresso} (QE) and \textsc{Octopus}. 
For this purpose we chose observables such as molecular bond lengths and crystal lattice parameters based on total-energy minimization, as well as linear optical response from real-time propagation as the relevant quantities to be compared. 
For details on calculation parameters please refer to Ref.~\cite{Andrade2021}. 
As apparent from the comparisons shown in Figure~\ref{fig:inq_results}, we find that absolute total energies from \textsc{inq} are within \(0.3~\mathrm{mHartree/atom}\) of the QE reference. 
Furthermore, the optical response of a gas phase water molecule as obtained from \textsc{inq} is seen to be in excellent agreement with the corresponding result from \textsc{octopus} in the time domain (Fig.~\ref{fig:inq_results}(c)) and therefore excitation frequ1encies in the optical absorption spectrum are also found to match closely across a wide energy range. 

\textbf{Mention stopping, hook to Future work}

\subsubsection{Tight Binding model for nonequilibrium dynamics}\label{sec:tight-binding}

Within NPNEQ, \emph{ab initio} code development efforts lead by the LLNL and SLAC teams are complemented by efforts at LBNL to develop tight-binding models for non-equilibrium dynamics with the recognition that feedback between qualitative models and quantitative simulations is broadly beneficial for knowledge generation within materials science. 
In particular, careful model building based on first-principles calculations combined with large-scale model simulations is a promising route to produce interesting insights into the dynamic behavior of  materials driven far from equilibrium. 
We have developed an atomic-orbital-based tight-binding code for simulations on large time and length scales to supplement the capabilities of \textsc{inq}. 
These models treat charge, spin, orbital, and atomic degrees of freedom explicitly, capturing the quantum effects of charge carriers, the non-collinear spin dynamics, and the cooperative lattice vibrations. 

With the current tight-binding code, structural and electronic optimization can be performed to study ground state properties. 
For simulating relaxation dynamics of excited solids, the code supports Ehrenfest dynamics and Car-Parrinello dynamics. 
The effect of linearly polarized light field is implemented via the Peierls substitution method. 
Our recently published work on manganites~\cite{Rajpurohit2021} demonstrates the potential of such tight-binding models to study non-equilibrium dynamics in functional quantum materials. 

During the project period, the code has been extended to simulate multilayer oxide structures, layered transition metal dichalcogenides (TMDC), and circularly polarized excitations. 
This has led to multiple scientific outputs and insights both through intra-NPNEQ~\cite{Rajpurohit2021} and external collaborations~\cite{Siddiqui2020} as outlined in the next subsection. 
Work is in progress to implement further extensions such as spin-orbit couplings, van der Waals (vdW) interactions, and Nose-Hover thermostat to incorporate finite-temperature effects. 
Additionally, the inclusion of external magnetic fields will allow for the simulation of magnetic excitations and magnon dispersion. 
Solving for these spin-waves will be done in large unit cells using an effective spin Hamiltonian consisting of Zeeman effects, exchange interactions and magnetic
anisotropies. 

\subsection{Collaborative science}

\textcolor{red}{Descriptions of type-A projects.}

\subsubsection{Role of nonequilibrium spin-phonon coupling in enhancing bulk photovoltaic effect}\label{sec:BPVE}

During the project period, our work on the materials system of \(\mathrm{Pr_xCa_{1-x}MnO_3}\) has resulted in the discovery of a new way to enhance the bulk photovoltaic effect by optically induced phase transitions. 
These results, currently under review, are reported in the preprint Ref.~\cite{Rajpurohit2021}. 
Our study provides a unified picture of the photocurrent generation and its evolution through real-time simulations, and will have a significant impact in field of the bulk photovoltaic effect (BPVE) where understanding, predicting, and ultimately controlling the photovoltaic behavior remains a challenge.
We show theoretically that this strongly correlated oxide perovskite displays strong photocurrent enhancement when it is driven into a hidden magnetic phase transition.
This phenomenon is highly non-perturbative, and lies outside the perturbative paradigm of shift and ballistic currents used in this field. 
The maximum photoresponsivity obtained is almost an order of magnitude higher than that reported for other transition metal oxides such as \(\mathrm{BiFeO_3}\) and \(\mathrm{BaTiO_3}\).

\begin{figure}[ht]
	\centering\includegraphics[width=1.0\linewidth]{figures/photocurrent_old}
	\caption{
		Simulated photocurrent of \(\mathrm{Pr_xCa_{1-x}MnO_3}\), as a function of light intensity and time after excitation. 
		Rapid enhancement of photocurrent (left) in indicative of a nonperturbative process beyond the frameworks of previously studied shift and ballistic current theories.
	}
	\label{fig:PCMO}
\end{figure}

To study this effect, we use a methodology based on real-time simulations of a first-principles based Hubbard model which treats all the relevant degrees of freedom, namely charge, spin, orbital and lattice, explicitly. 
Our methodology allows us to study transient effects such as THz oscillations and their decay on picosecond timescales, as well as current saturation at higher light intensities, all of which are inaccessible using perturbative frequency domain techniques that have been in use for BPVE. 

In this work, {\bf Tan} has supervised photocurrent simulations, {\bf Pemmaraju} has provided input on the handling of dissipative processes, and {\bf Ogitsu} has guided the research focus to questions that can be answered in greater detail using the \textsc{inq} code in the remaining project period. 
Follow-up work will utilize the exact-exchange functional features of the \textsc{inq} code to explore the impact of nonlocal exchange on photocurrents in these strongly correlated systems. 
GPU optimization of the \textsc{inq} code will be used to extend these simulations to the relevant time scales identified above.  

\subsubsection{Halide perovskite work}

High defect tolerance has been considered as a main reason for the long charge carrier lifetime and high photoluminescence quantum yield in bulk lead halide perovskites (LHPs). 
On the other hand, surface defects play a critical role in determining charge carrier dynamics and optical properties, especially for LHP nanocrystals and quantum dots. Revealing the nature of surface defects and developing strategy for their effective passivation are thus of strong interest. 
{\bf Tan} and {\bf Ogitsu}, together with collaborators at UCSC~\cite{Smart2021}, have used first-principles calculations to reveal that interstitial and antisite defects in a prototypical LHP, \(\mathrm{CsPbBr_3}\), can have lower formation energy when they form at the surface instead of bulk while simultaneously creating deep trap states within the bandgap.
We show how to choose molecular ligands to passivate these defects, eliminating trap states in favor of shallow states, which enhances photoluminescence. 
(See Fig.~\ref{fig:defect}) This work has implications for  the field of LHPs, especially LHP devices where interfacial chemistry plays an important role in the optoelectronic properties and stability.

This work forms the basis for planned real-time simulations of lossy processes at these defects in the LHPs, such as non-radiative recombination processes and dephasing, which are of fundamental interest for optoelectronic applications. 
Even though typical RT-TDDFT simulations are shorter than the natural time scales of these processes, it has been demonstrated that their rates can still be extracted, by using a cumulant expansion approach~\cite{Qiao2020}. 
Scalability of the \textsc{inq} code will enable these calculations in challenging systems such as LHP surfaces defects, which contain a large number of electrons.     

\begin{figure}[h]
	\centering
	\includegraphics{figures/defect_passivation}
	\caption{
		Schematic atomic structure of defects at the surface of \(\mathrm{CsPbBr_3}\), which are susceptible to non-radiative recombination when unpassivated (left), but become less detrimental to radiative recombination when passivated (right).
	}
	\label{fig:defect}
\end{figure}

\textcolor{red}{Descriptions of selected type-B projects.}

\subsubsection{Ultrafast dynamics in strong-field ionized liquid water} 

\begin{figure}[h]
	\centering
	\includegraphics{figures/Water}
	\caption{
		The ionization of liquid water and formation of solvated \(\mathrm{H_3O^+}\), \(\mathrm{OH\cdot}\) radical and electron species is followed by the local thermalization involving energy exchange between the electronic and ionic degrees of freedom.
	}
	\label{fig:water}
\end{figure}

Photo-excitation induced structural dynamics in liquid water represents a paradigmatic case of excited state electron-ion dynamics in the condensed phase. 
In particular, the radiolysis of liquid water and its aftermath is a process of fundamental importance in a variety of technological contexts~\cite{Garrett2005} such as energy harvesting, biochemistry, biomedicine, corrosion mitigation, etc. 
Therefore, investigating the interplay between the electronic and ionic degrees of freedom in photo-ionized liquid water on the femto-picosecond timescales relevant to the creation and transformation of short-lived reaction intermediates~\cite{Loh2020} is of interest to both experiment and theory.
Within the current funding period, NPNEQ researchers at SLAC collaborated with experimental teams working at SLAC's MeV-UED instrument to uncover the structural finger-prints of hydronium cation-hydroxyl radical pairs on femtosecond timescales in strong-field ionized liquid water (See Fig.~\ref{fig:water}). 
In this study, we deployed velocity-gauge real-time TDDFT as implemented within the \textsc{salmon} code~\cite{salmon} to estimate the photo-ionization fraction in liquid water for experimentally relevant laser pulse parameters. 
Our simulations revealed a nonlinear multi-step photo-ionization process in liquid water subjected to high-intensity near-infrared radiation and our predicted excited electron energy distribution subsequently informed classical models of thermalization based on molecular dynamics. 
A manuscript summarizing the findings of our experiment-theory collaboration is currently under review at the journal \emph{Science}.
In the near future we plan to further extend our studies in this prototypical water system by simulating nonadiabatic Ehrenfest electron-ion dynamics during and after laser illumination using our RT-TDDFT platform \textsc{inq} and compare our results with recent ultrafast experiments. 

%\subsubsection{WTe2 collaborations}

\subsubsection{Charge density wave melting in 2D materials}

In this experimental collaboration~\cite{Siddiqui2020}, we perform the first ultrafast investigation of \(\mathrm{TaTe_2}\), which exhibits unique charge density wave (CDW) and lattice structural order characterised by a transition upon cooling from stripe-like trimer chains into a (\(3\times 3\)) superstructure of trimer clusters. 

We develop computational models of the ultrafast photoinduced dynamics in \(\mathrm{TaTe_2}\), incorporating  electronic and atomic structure and their strong interactions to calculate the atomic trajectories. 
Density-functional calculations indicate that the initial quench is triggered by Ta trimer bonding to nonbonding transitions that destabilises the clusters, unlike CDW melting in other \(\mathrm{TaX_2}\) compounds. 
These predictions were verified by experimental work utilising MeV-scale ultrafast electron diffraction to resolve structural dynamics following intense pulsed laser excitation. 
We observe a rapid \(1.4~\mathrm{ps}\) melting of the low-temperature ordered state, followed by recovery of the clusters via thermalization into a hot superstructure persisting for extended times.
Our work paves the way for further exploration and ultimately directed manipulation of the trimer superstructure for novel applications.

\section{Future Work}

\subsection{Code developments}
Having validated the modern C++ based design and functionality of \textsc{INQ} and demonstrated parallelism over hybrid i 
\subsubsection{Functionality additions to INQ}

\paragraph{Exact exchange and Axc}

Nonlocal functionals for Generalized Kohn-Sham (GKS) RT-TDDFT (Task 1.1)
% The need for more accurate descriptions of the electron density and its polarization response to perturbations across length scales is critical for real-time explorations of ultrafast dynamics. Field-driven excitations relevant to optical transitions require accurate band gaps and excited state evolution113, 116, 118. The potential to form self-trapped excited states, through electron-nuclear interactions, such as exciton-polaritons, places stringent accuracy requirements on exactly those areas where traditional (semi-)local exchange-correlation functionals fail due to significant self-interaction errors that limit or prevent electronic localization and the associated symmetry-breaking forces on local regions of the ionic lattice/nuclear coordinates112, 116, 144. 
% Nonlocal functionals featuring a fraction of Fock exchange provide one solution to this problem, generally reducing self-interaction, increasing bandwidth in the electronic DOS and reducing polarizabilities to values closer to experimental estimates118, 119. However, evaluation of the Fock operator within the exact-exchange formalism presents significant computational overhead for static calculations and can render time-dependent studies intractable. Such calculations, within a plane-wave basis (as proposed in this effort) present costs that scale as Ng log(Ng) Ne2, where Ng, the number of plane-wave basis functions can easily approach millions for large system sizes, even if the number of electrons, Ne, is only in the thousands. In partnership with NERSC, PI Prendergast recently succeeded in providing order of magnitude speed-up within the hybrid exact-exchange implementation for the open-source code, PWscf (part of the QuantumESPRESSO suite145) through (1) reorganization of the data layout when switching between the external local DFT self-consistent field loop and an internal non-local Fock matrix evaluation loop and (2) parallelization of the Fock matrix evaluation via band pairing146. These changes have been implemented within PWscf since version 6.1 of QuantumESPRESSO and recent applications include studies of transition metal oxides: variations in electronic DOS upon lithiation147 and significant improvements in transition state energetics for lithium ion transport148. 
% Furthermore, for large scale plane-wave applications, we will require convergent approximations to the Fock operator with further reductions in computational overhead, such as the Adaptively Compressed Exchange (ACE) approach of Lin149. In this development, the full exchange operator is replaced with a lower rank (Ne vs. Ng) approximant with convergence constraints. This has also been implemented with QuantumESPRESSO (since version 6.1) and another plane-wave implementation in the Discontinuous Galerkin DFT code (DGDFT150) and has been applied in benchmark studies of bulk silicon supercells and to the study of water adsorption on silicene151. ACE contrasts with previous efforts to exploit the locality of exchange interactions through transformations of the occupied Kohn-Sham orbital subspace to a basis of localized functions, such as maximally-localized Wannier functions (as developed by Car et al. 152 in the CP code within QuantumESPRESSO) or via recursive subspace bisection (as developed by Gygi et al. 153 in the Qbox code).
% The ACE approach for hybrid exact-exchange, combined with the parallel-transport (PT) gauge discussed previously represents a promising avenue (PT-ACE) towards efficient plane-wave basis set RT-TDDFT simulations133 within the Generalized Kohn-Sham (GKS) scheme. It is noted however by Jia et al133, that 1024 silicon atoms, running for ~30 fs employing PT-ACE still requires 1 week to compute on 2048 cores of Edison (at NERSC), although without PT-ACE, it might take a year to do the same. Therefore, there is clearly work still to be done to both improve the strong scaling of plane-wave implementations of hybrid exact-exchange within RT-TDDFT, by adapting promising algorithms like PT-ACE for scalable GPU-based architectures such as within Qbox. To this end, the simplifications and algorithmic improvements outlined above highlight potential opportunities to offload significant computational work while leveraging hardware specific implementations of standard matrix-multiply and fast Fourier transform routines. 
% (Task 1.1) In view of the above, the NPNEQ team will develop GKS RT-TDDFT functionality featuring range-separated hybrid XC functionals by leveraging  algorithms such as PT-ACE within the scalable Qbox framework.  


\paragraph{Non-collinear magnetism and spin-orbit}
Spin-orbit coupling for dynamics (Task 1.3)
 
% In time-independent problems the description of spin-orbit coupling is necessary to reproduce accurate atomic energies and level splitting. In thermal equilibrium the spin state and the existence of magnetism is determined by free energy minimization. In time-dependent problems the situation is more complicated, since the only way to change the spin state is by an external magnetic field or internally by spin-orbit coupling. Processes like ultrafast demagnetization can be phenomenologically explained by models that assume kinetics upon free energy models 162 but a practical microscopic model is still absent.
% Specifically, we will implement spin-orbit coupling (SOC) dynamics, which will significantly improve fidelity of quantum dynamics simulations of functional materials that often comprise transition metals and exhibit complex magnetism. Our implementation of SOC for the planewave basis will not use the projection approximation that virtually all the currently existing (TD)DFT software relies on; as it has been shown to be inappropriate for a description of time dependent quantum mechanical evolution 163.
% The ubiquitous projection approximation allows to use a natural representation of the spin-orbit interaction by an additive term of the type  with prefactors calculated at the atomic (pseudopotential) level (   is a local projection of the electronic orbital angular momentum). The definition of  depends on ionic positions, which are varying dynamically along with the simulation. Effectively this defines a representation in a curved and changing Hilbert space that is very complex to code163 .
% Although this picture of SO coupling it is usually justified by the dominance of localized orbitals, in the non-adiabatic context, the simplifications of the projection method carry a heavy complexity cost. Additionally, a time-dependent theory of magnetism is necessarily non-colinear, in order to continuously connect different spin configurations.
% Time-dependency, non-colinear magnetism and dynamical ions (coupled to both electron-orbital and spin) call for a new approach to the problem of spin in first-principles simulations. Our alternative strategy starts from the the Breit-Pauli Hamiltonian in the two component spinor algebra 164. The Breit-Pauli terms incorporate scalar relativistic corrections to the non-relativistic Hamiltonian, and contain spin-explicit terms describing the interaction between the orbital magnetic moments and spin magnetic moments:

% where  is the gradient (force) of the nuclear potential. This part of the Hamiltonian contains two-body terms that are unacceptable in the Kohn-Sham theory, therefore we will use the gradient of self-consistent effective potential as presented by Krieger et al.
% This representation of the SO is independent on the labeling of ions and not-explicitly dependent on the ion positions (only through the potential); it doesn’t rely on a local projection of the angular momentum and it is more suitable for a time dependent treatment with co-evolving ions. In the same way that a quantum ion-electron interaction leads to advanced picture of electron-phonon coupling (a two-temperature system),128 this development will lead to a fuller picture to guide advances of effective models of spin-lattice-electron dynamics166. This is important, in particular, to handle time evolution of Weyl points driven by lattice motion or circularly polarized light, addressing the needs for modeling experimental efforts described in other sections. 

\paragraph{Spiral boundary conditions}

\paragraph{Beyond Ehrenfest dynamics} 
Mention: Electron-phonon development from stopping power.  Figure.

\paragraph{Interfacing \textsc{inq} with Tight-Binding codes}

We plan to use \textsc{inq} in combination with our tight-binding dynamics code (Section.~\ref{sec:tight-binding}) for parameter extraction and for identification of the relevant degrees of freedom to be retained in these downfolded models. 
Having a well-defined interface for these tasks will accelerate projects of the sort described in Sections~\ref{sec:BPVE},~\ref{sec:tight-binding}. Dynamical Wannierization along TDDFT trajectories will provide Hamiltonian matrix elements for the construction of these models, generalizing an approach previously taken for Wannierization along classical MD trajectories~\cite{Abramovitch2021}. 
These script-heavy workflows will be greatly simplified in \textsc{inq}, which will include a language-level interface to the \textsc{Wannier90} library~\cite{Mostofi2008}. 
Other types of script-heavy workflows, especially those with high throughput aspects, will benefit similarly from the modularity of \textsc{inq}, which makes scripting and post-processing obsolete in favor of programmability within the single-language \textsc{inq} framework.

Scientifically, having the \textsc{inq} code coupled to local-basis models will enable the exploration of research questions relating to the dynamics of localization. 
The appropriate parameterization of tight-binding models suitable for describing non-adiabatic dynamics remains an open question, and similar questions remain in the propagation of local orbitals, particularly close to metallicity~\cite{Yost2019}. 
With these tools, we will study how orbital localization arises from decoherence, finite-temperature effects, and scattering processes.  

\subsection{Science Applications}

Several ongoing projects involving close collaboration between experimental efforts and theory from this team are in progress.

\subsubsection{Spin dynamics in layered antiferromagnets}\label{sec:2dafm}

In collaboration with {\bf Tan} and {\bf Pemmaraju}, {\bf Lindenberg} has been investigating the ultrafast dynamical response of two-dimensional antiferromagnetic materials.
In antiferromagnets, electron exchange interactions result in antiparallel or non-collinear microscopic spin correlations with negligible macroscopic magnetization.
Given their low-loss, intrinsic THz frequencies, and insensitivity to stray fields, antiferromagnetic spintronics holds great potential in realizing high-speed communications and new types of information storage technologies. 
Atomically-thin van der Waals crystals like \(\mathrm{FePS_3}\), \(\mathrm{NiPS_3}\) and \(\mathrm{MnPS_3}\) represent a new type of antiferromagnetic materials class with strong 2D quantum confinement and the ability to tune this functionality at high speed and with low energy cost taking advantage of the weak interlayer van der Waals bond.
In initial work we have carried out time domain THz emission spectroscopy probing the ultrafast dynamics of the magnetization in exfoliated flakes of \(\mathrm{NiPS_3}\).
In these experiments an ultrafast optical pump pulse excites above band gap at \(400~\mathrm{nm}\) to create free carriers which then couple to the intrinsic magnetization of the material.
By detecting the emitted THz fields from the sample, we probe the time-dependent free (associated with the flow of electrons) and bound (associated with the time-dependent magnetization) currents.
Following directly from Maxwell’s equations one finds: 

\begin{equation}
\mathbf{E}_\text{THz} \propto \frac{\partial\mathbf{J}}{\partial t} + \frac{\partial}{\partial t} (\nabla \times \mathbf{M})
\end{equation}
where \(\mathbf{J}\) is the free current and \(\mathbf{M}\) is magnetization. 
Fig.~\ref{fig:NiPS3} shows the experimentally measured THz waveform and the peak-to-peak emission amplitude as a function of temperature, showing an enhancement at the Neel temperature.
This demonstrates the sensitivity of the measurement to the intrinsic antiferromagnetic dynamics.
Further, by measuring the polarization state of the emitted THz fields one can extract information about the crystallographic directions of the associated currents.
We find there are two contributions to the measured fields, involving a current normal to the sample surface associated with the separation of electrons and holes at the surface superposed on a bound current likely associated with an induced rotation of the initially in-plane antiferromagnetic magnetization state.
Further spectral analysis of the emitted fields may allow us to extract information about particular magnons associated with the light-induced response.
This work opens up new possibilities for controlling the properties of antiferromagnetic materials on ultrafast time-scales. 
Ongoing work additionally involves upcoming experiments at the SLAC National Accelerator Laboratory Ultrafast Electron Diffraction facility where we will in parallel investigate the correlated structural dynamics under similar photoexcitation conditions.

\begin{figure}[ht]
	\centering\includegraphics[width=1.0\linewidth]{figures/NiPS3}
	\caption{
		(left)  Measured THz electric field waveform emitted from \(\mathrm{NiPS_3}\) antiferromagnetic sample under femtosecond \(400~\mathrm{nm}\) excitation.
		(right) Peak THz emission amplitude as a function of temperature, showing strong enhancement at the Neel temperature (\(155~\mathrm{K}\)), showing direct sensitivity to time-dependent magnetization.
	}
	\label{fig:NiPS3}
\end{figure}

In conjunction with this experimental work, {\bf Tan} and {\bf Pemmaraju} are investigating a number of theoretical approaches to this problem.
First, static, first-principles density functional theory are used to solve for the electronic structure, phononic properties, and electron-phonon interactions of the \(\mathrm{NiPS_3}\) materials system. 
Second, the RT-TDDFT approach will be used to directly simulate density and spin fluctuations in time using the \textsc{inq} code.
The evolution of the calculated diffraction intensities of the spin, charge and orbital patterns will provide dynamics of the order parameters to assist the experimental time-resolved diffraction study.
Trajectories will be examined to understand the structural and electronic mechanisms behind the evolution of spin order peaks, and their associated time scales. 
Finally, the RT-TDDFT simulations will provide data for parameterization of a model for long-wavelength magnon simulations. 
This semi-classical approach involves the construction of an effective spin Hamiltonian consisting of Zeeman, exchange and anisotropic magnetic interactions, with parameters extracted from the \textsc{inq} code. 
This Hamiltonian will be solved for large magnetic supercell to produce the spin-wave dispersions. 
Development of this model will be accelerated by the team's recent experience with similar models for 2D TMDCs and manganites~\cite{Siddiqui2020, Rajpurohit2020}. 
As a by-product of this work, the \textsc{inq} code will contain an interface that will facilitate user extraction of parameters for model simulations.

\subsubsection{Demagnetization in optically excited perovskite nickelates}

The perovskite nickelates \(\mathrm{RNiO_3}\) are host to a number of unique structural, electronic and magnetic transitions~\cite{Giovannetti2009,Hampel2017,Beyerlein2018}.
The melting of the low-temperature antiferromagentic order in photoexcited nickelates~\cite{Caviglia2013} is suggested to be directly linked to the melting of charge order, but the details and control of this coupling are not well understood. 
The metal-insulator transition involves structural changes that split \(\mathrm{Ni^{3+}}\) ions to two inequivalent sites, which surprisingly do not involve any Jahn-Teller distortion, unlike typical charge-transfer transitions in other oxide perovskites. 
In future work, we plan to address these unresolved questions using a combination of real-time TDDFT simulation ({\bf Tan}) and time domain THz emission spectroscopy together with Ultrafast Electron Diffraction ({\bf Lindenberg}). 
Similar to the approach to be taken in Section~\ref{sec:2dafm}, this combination of experimental techniques will measure antiferromagnetic dynamics together with correlated structural changes. 
Simulation efforts will target the process of charge order melting due to the instantaneous redistribution of charge carriers during photoexcitation process, and the creation of metastable spiral spin structures over \(0.5~\mathrm{ps}\). 
\textsc{inq} scaling will let us extend these calculations to heterostructures, which are of interest because of possible superconductivity in these structures. These simulations will investigate exchange bias effects at the \(\mathrm{RMnO_3}/\mathrm{RNiO_3}\) interface, the modulation of orbital occupation due to strain and interface effects, and energy transfer across the interface during and after photoexcitation.

\subsubsection{Nonlinear nonequilibrium response in topological materials}

Nonlinear optical phenomena arise when light at sufficiently high intensities interacts with matter.
Nonlinear response underlies technologically important physical effects including shift-currents, harmonic generation and frequency upconversion in photovoltaic and optoelectronics applications among others.
Recent insights that nonlinear responses are intimately tied to the fundamental topological properties of the quantum materials, has added to the interest in controlling the excited states of these systems with ultrafast pump-probe and nonlinear multidimensional spectroscopies.

The computational studies proposed in this project are inspired by recent joint experiment-theory results reported by PIs {\bf Lindenberg} and {\bf Pemmaraju} on ultrafast nonlinear responses in the Weyl semimetal \(\mathrm{WTe_2}\)~\cite{Xiao_2020}. 
Our studies showed that terahertz frequency light pulses induce large amplitude interlayer shear strains in \(\mathrm{WTe_2}\), modulating its symmetry and potentially its topological phase. 
The THz induced shear strains in \(\mathrm{WTe_2}\) were also found to be associated with a large transient modulation of the second harmonic generation (SHG) response whereby in the THz driven sample, the SHG response is dramatically reduced compared to noncentrosymmetic bulk \(\mathrm{WTe_2}\). 
However since the low order nonlinear response in materials is fundamentally connected with topological quantities such as the Berry connection and Berry curvature, the question remains as to whether the observed transient reduction in SHG is a signature of a topological phase transition involving the annihilation of Weyl nodes in the Bloch band structure or that of a structural transition to a centrosymmetric geometry.

We will investigate the electronic \emph{vs.} structural origins of recently observed large transient changes in the second harmonic response of the Weyl semimetal \(\mathrm{WTe_2}\) in response to terahertz driven lattice modulations.
The relationship between symmetry, topology and SHG response in \(\mathrm{WTe_2}\) will be investigated using both frequency and time-domain approaches. Simulations of SHG response in \textsc{inq} will go beyond the Born-Oppenheimer approximation to model intrinsic non-equilibrium states involving transient field-driven modulations of the electrons and holes near the Weyl band crossing points.
Utilizing the Ehrenfest dynamics capability of the \textsc{inq} code, we plan to investigate the role of transient in-plane currents in driving the observed \(0.24~\mathrm{THz}\) shear-shear responses in \(\mathrm{WTe_2}\), and the role of electron doping in tuning these responses. 
Such studies will contribute significantly to a detailed material specific understanding of ultrafast nonlinear response in topological materials within the TMDC family and beyond.

\clearpage

